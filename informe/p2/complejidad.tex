\subsubsection{Pseudocódigo}
\begin{algorithm}[h!]
\caption{Estructura del algoritmo Goloso}
\begin{algorithmic}[1]
	\Function{golosa}{}
	\While{haya enemigos}
	 \state {tomo el primer enemigo vivo}
	 \while{haya enemigos restantes y tirar en ese enemigo mate al primer enemigo vivo}
	  \state {avanzo al siguiente enemigo}
	  \EndWhile
	  \state {guardo el ultimo enemigo q terminaria matando al primero vivo si le tiro la genkidama}

		\While{hay enemigos restantes y tirar en ese enemigo mata al ultimo enemigo que mate al primer enemigo vivo}
		\state {avanzo al siguiente enemigo}
        \EndWhile
	\EndWhile
	\EndFunction
\end{algorithmic}
\end{algorithm}

\subsubsection{Analisis}
El algoritmo tiene un ciclo principal con dos ciclos internos,  para cada enemigo, o bien es tomado en cuenta en el primer ciclo o bien en el segundo, salvo los casos donde el enemigo no mate al primer enemigo vivo ni este sea matado por el ultimo enemigo que mata al primer enemigo vivo.

Para la complejidad del algoritmo, consideramos el mejor caso el caso en que si tiramos la genkidama en cualquiera de los puntos, termina matando a todos los enemigos. y daria una complejidad de $O(N)$
El Peor caso seeria si tirar una genkidama en cualquier enemigo termine matando a ese enemigo y solo a ese enemigo, en ese caso cada enemigo termina siendo tomado en cuenta en ambos ciclos internos y en el ciclo principal, dando una complejidad de $O(3N) = O(N)$
Y el caso promedio que es simplemente aleatorio, consideramos tambien que esta en el orden de $O(N)$