\documentclass[a4paper, 12pt, spanish]{article}

\usepackage[paper=a4paper, left=1.5cm, right=1.5cm, bottom=1.5cm, top=3.5cm]{geometry}
\usepackage[spanish, es-noshorthands]{babel}
\usepackage[utf8]{inputenc}
\usepackage[none]{hyphenat}
\usepackage[colorlinks,citecolor=black,filecolor=black,linkcolor=black,    urlcolor=black]{hyperref}

% Simbolos matemáticos
\usepackage{amsmath}
\usepackage{amsfonts}
\usepackage{amssymb}
\usepackage{algorithm}
\usepackage[noend]{algpseudocode}
\usepackage{algorithmicx}
\usepackage{listings}

% Descoración y gráficos
\usepackage{caratula}
\usepackage{graphicx} 
\usepackage{fancyhdr}
\usepackage{lastpage}
\usepackage{caption}
\usepackage{subcaption}
\usepackage{multirow}
\usepackage{alltt}
\usepackage{tikz}
\usepackage{color}
\usepackage{gnuplottex}

% Acomodo fancyhdr.
\pagestyle{fancy}
\thispagestyle{fancy}
\addtolength{\headheight}{1pt}
\lhead{Algoritmos y Estructuras de Datos III}
\rhead{$1^{\mathrm{do}}$ cuatrimestre de 2016}
\cfoot{\thepage /\pageref*{LastPage}}
\renewcommand{\footrulewidth}{0.4pt}

\floatname{algorithm}{Pseudocódigo}
\algrenewcommand\algorithmicfunction{\textbf{Función}}
\algrenewcommand\algorithmicfor{\textbf{para}}
\algrenewcommand\algorithmicdo{\textbf{hacer:}}
\algrenewcommand\algorithmicif{\textbf{si}}
\algrenewcommand\algorithmicthen{\textbf{entonces:}}
\algrenewcommand\algorithmicelse{\textbf{si no:}}
\algrenewcommand\algorithmicend{\textbf{fin}}
\algrenewcommand\algorithmicreturn{\textbf{devolver}}

\sloppy

\parskip=5pt % 10pt es el tama de fuente

% Pongo en 0 la distancia extra entre itemes.
\let\olditemize\itemize
\def\itemize{\olditemize\itemsep=0pt}


\materia{Algoritmos y Estructuras de Datos III}
\grupo{Conformación del grupo}
\tituloCaratula{Trabajo Práctico N$^\circ$1}

\integrante{Fernando Abelini}{544/09}{ferabelini@outlook.com}
\integrante{Jose Fernando Alvaro}{89/10}{fer1578@gmail.com}
\integrante{Uriel Jonathan Rozenberg}{838/12}{urielrozenberg@hotmail.com}
\integrante{Roman Litvin}{183/14}{rglitvin@gmail.com}

\usepackage{tikz}
%\usepackage{tikz-qtree}


\usetikzlibrary{arrows,backgrounds,calc}

\pgfdeclarelayer{background}
\pgfsetlayers{background,main}



\newcommand{\convexpath}[2]{
[ 
    create hullnodes/.code={
        \global\edef\namelist{#1}
        \foreach [count=\counter] \nodename in \namelist {
            \global\edef\numberofnodes{\counter}
            \node at (\nodename) [draw=none,name=hullnode\counter] {};
        }
        \node at (hullnode\numberofnodes) [name=hullnode0,draw=none] {};
        \pgfmathtruncatemacro\lastnumber{\numberofnodes+1}
        \node at (hullnode1) [name=hullnode\lastnumber,draw=none] {};
    },
    create hullnodes
]
($(hullnode1)!#2!-90:(hullnode0)$)
\foreach [
    evaluate=\currentnode as \previousnode using \currentnode-1,
    evaluate=\currentnode as \nextnode using \currentnode+1
    ] \currentnode in {1,...,\numberofnodes} {
-- ($(hullnode\currentnode)!#2!-90:(hullnode\previousnode)$)
  let \p1 = ($(hullnode\currentnode)!#2!-90:(hullnode\previousnode) - (hullnode\currentnode)$),
    \n1 = {atan2(\x1,\y1)},
    \p2 = ($(hullnode\currentnode)!#2!90:(hullnode\nextnode) - (hullnode\currentnode)$),
    \n2 = {atan2(\x2,\y2)},
    \n{delta} = {-Mod(\n1-\n2,360)}
  in 
    {arc [start angle=\n1, delta angle=\n{delta}, radius=#2]}
}
-- cycle
}

\newcommand{\todo}[1]{
\textbf{\color{red}{\underline{Nota:} #1}}
}

\newcommand\param[3]{\ensuremath{\mathbf{\textbf{#1}}\,#2\!:} \texttt{#3}}

\let\state\State
\let\while\While
\let\endwhile\EndWhile
\let\endif\EndIf
\let\elseif\ElsIf
\let\for\For
\let\endfor\EndFor
\let\function\Function
\let\endfunction\EndFunction


\begin{document}

%\setcounter{tocdepth}{2}

\begin{titlepage}

\maketitle

\end{titlepage}

\tableofcontents

\newpage
\section{Problema 1:  Kaio Ken}
\subsection{Introducción}
\subsubsection{Explicación del problema}
El problema consiste en destruir a los androides del doctor Maki Gero mediante la menor cantidad posible de Kamehamehas lanzados por Goku.
A fin de resolver este problema modelamos a los androides como puntos en el plano y los Kamehameha como semirrectas en el mismo. Dicho es el problema se reduce a atravesar todos los puntos del plano con la menor cantidad de semirrectas.

\subsubsection{Ideas e implementación}
Para resolver el problema usaremos la técnica de backtracking. Mediante el mismo iremos eligiendo de a 2 elementos dentro de un vector con las posiciones de los androides y calculando la pendiente de la recta que los une nos fijaremos si atraviesa posiciones de otros androides en su recorrido. Marcaremos todos estos puntos como androides destruidos.
Una posible solución es encontrada cuando no quedan androides por destruir.
La estructura del algoritmo de backtracking es la siguiente:

\begin{algorithm}[h!]
\caption{Estructura del algoritmo de Backtracking}
\begin{algorithmic}[1]
	\Function{Backtrack}{}
	\If{no quedan mas androides por destruir}
		\State Mejor solución encontrada hasta ahora
		\State Guardo el vector de androides como mejor solución
	\Else
		\For{cada par de androides $a_1$, $a_2$}
            \If{$a_1$ y $a_2$ no estan destruidos}
                \If{la solución parcial es menor que la mejor solución +1}
                    \State Destruir los androides $a_1$, $a_2$
                    \State Destruir los androides en la semirrecta entre $a_1$ y $a_2$
                    \State Backtrack()
                    \State Restaurar los androides en la semirrecta entre $a_1$ y $a_2$
                    \State Restaurar los androides $a_1$, $a_2$                        
                \EndIf
            \EndIf
		\EndFor
	\EndIf
	\EndFunction
\end{algorithmic}
\end{algorithm}

\subsubsection{Podas}
En esta sección se explican las podas implementadas.

\begin{enumerate}
\item Una de las primeras podas que aplicamos es la de verificar si el androide no se encuentra ya destruido para no seguir por esa rama.
\item Aplicamos otra poda comparando que los 2 androides a unir con la semirecta no sean el mismo.
\item Aplicamos también una poda al comparar la solución parcial con la mejor solución encontrada hasta ese momento y solo seguimos en caso que sea posible mejorar la mejor solución encontrada hasta ese momento.
\end{enumerate}


\subsection{Correctitud}
Para probar la correctitud de este algoritmo hay qur probar dos cosas:
1.que cubre todos los puntos
2.que es el resultado minimo:


1.Esta demostracion es trivial, es parte del algoritmo  calcular que se cubran todos los puntos
2.lo hacemos por induccion sobre la cantidad de puntos.

La hipotesis inductiva es que para N puntos el algoritmo es minimo

Caso base:
numero de puntos = 1:
Esta demostracion es trivial. el algoritmo nos muestra que tira la genkidama en ese unico punto y no se pueden tirar menos genkidamas

Caso N+1:
Dada  la HI, si agregamos un punto mas respetando las condiciones nos quedarian tres casos:
1.O bien la solucion planteada para los N primeros puntos alcanza para que muera el N+1 enemigo, en ese caso seguiria siendo minimo
2.O bien si la genkidama se tiraba en el punto N, y tirarla en N+1 mataria a N, entonces  en vez de tirarse en N se tira en N+1, nuevamente dando minimo
3.O bien el punto esta lo suficientemente lejos del resto de los puntos como para que la unica forma de matar a ese enemigo sea tirandole una genkidama a ese punto y dicha genkidama no mate a nadie, en ese caso volveria a ser un minimo por que se requeriria si o si una genkidama mas para llegar a ese punto
\subsection{Complejidad}
\subsubsection{Pseudocódigo}

\subsubsection{Analisis}


\subsection{Analisis experimental}
Para la realización de los tests nos concentraremos en confirmar la cota para el peor caso, el mejor caso y el comportamiento del algoritmo para casos aleatorios.

\begin{itemize}
\item El mejor caso se da cuando todos los androides están alineados en una misma fila, columna o diagonal, bajo este supuesto basta un solo kamehameha para destruirlos.
\item El peor caso se da cuando los androides están ubicados en una posición tal que cualquier semirrecta que atraviese a un par de ellos no atravesara a ninguno del resto por lo que necesitaremos una cantidad de kamehamehas igual a la mitad de los androides para destruirlos a todos. Un caso como el mencionado se da si ubicamos a los androides sobre los puntos de una parábola de una función como puede ser una cuadrática por ejemplo.
\item El otro caso se da en aquellos que no caen en ninguno de los mencionados anteriormente sino que los androides son ubicados aleatoriamente en el plano.
\end{itemize}

A continuación se muestran los resultados generados.

\begin{figure}[h]
	\begin{center}
%		\resizebox{\columnwidth}{!}{\input{imgs/p3-mediciones.png}}
	\end{center}
	\caption{Mediciones de peor caso, mejor caso y caso promedio}
\end{figure}



\newpage
\section{Problema 2:  Genkidama}
\subsection{Introducción}
\subsubsection{Explicación del problema}
El problema consiste en destruir a los androides del doctor Maki Gero mediante la menor cantidad posible de Kamehamehas lanzados por Goku.
A fin de resolver este problema modelamos a los androides como puntos en el plano y los Kamehameha como semirrectas en el mismo. Dicho es el problema se reduce a atravesar todos los puntos del plano con la menor cantidad de semirrectas.

\subsubsection{Ideas e implementación}
Para resolver el problema usaremos la técnica de backtracking. Mediante el mismo iremos eligiendo de a 2 elementos dentro de un vector con las posiciones de los androides y calculando la pendiente de la recta que los une nos fijaremos si atraviesa posiciones de otros androides en su recorrido. Marcaremos todos estos puntos como androides destruidos.
Una posible solución es encontrada cuando no quedan androides por destruir.
La estructura del algoritmo de backtracking es la siguiente:

\begin{algorithm}[h!]
\caption{Estructura del algoritmo de Backtracking}
\begin{algorithmic}[1]
	\Function{Backtrack}{}
	\If{no quedan mas androides por destruir}
		\State Mejor solución encontrada hasta ahora
		\State Guardo el vector de androides como mejor solución
	\Else
		\For{cada par de androides $a_1$, $a_2$}
            \If{$a_1$ y $a_2$ no estan destruidos}
                \If{la solución parcial es menor que la mejor solución +1}
                    \State Destruir los androides $a_1$, $a_2$
                    \State Destruir los androides en la semirrecta entre $a_1$ y $a_2$
                    \State Backtrack()
                    \State Restaurar los androides en la semirrecta entre $a_1$ y $a_2$
                    \State Restaurar los androides $a_1$, $a_2$                        
                \EndIf
            \EndIf
		\EndFor
	\EndIf
	\EndFunction
\end{algorithmic}
\end{algorithm}

\subsubsection{Podas}
En esta sección se explican las podas implementadas.

\begin{enumerate}
\item Una de las primeras podas que aplicamos es la de verificar si el androide no se encuentra ya destruido para no seguir por esa rama.
\item Aplicamos otra poda comparando que los 2 androides a unir con la semirecta no sean el mismo.
\item Aplicamos también una poda al comparar la solución parcial con la mejor solución encontrada hasta ese momento y solo seguimos en caso que sea posible mejorar la mejor solución encontrada hasta ese momento.
\end{enumerate}


\subsection{Correctitud}
Para probar la correctitud de este algoritmo hay qur probar dos cosas:
1.que cubre todos los puntos
2.que es el resultado minimo:


1.Esta demostracion es trivial, es parte del algoritmo  calcular que se cubran todos los puntos
2.lo hacemos por induccion sobre la cantidad de puntos.

La hipotesis inductiva es que para N puntos el algoritmo es minimo

Caso base:
numero de puntos = 1:
Esta demostracion es trivial. el algoritmo nos muestra que tira la genkidama en ese unico punto y no se pueden tirar menos genkidamas

Caso N+1:
Dada  la HI, si agregamos un punto mas respetando las condiciones nos quedarian tres casos:
1.O bien la solucion planteada para los N primeros puntos alcanza para que muera el N+1 enemigo, en ese caso seguiria siendo minimo
2.O bien si la genkidama se tiraba en el punto N, y tirarla en N+1 mataria a N, entonces  en vez de tirarse en N se tira en N+1, nuevamente dando minimo
3.O bien el punto esta lo suficientemente lejos del resto de los puntos como para que la unica forma de matar a ese enemigo sea tirandole una genkidama a ese punto y dicha genkidama no mate a nadie, en ese caso volveria a ser un minimo por que se requeriria si o si una genkidama mas para llegar a ese punto
\subsection{Complejidad}
\subsubsection{Pseudocódigo}

\subsubsection{Analisis}


\subsection{Analisis experimental}
Para la realización de los tests nos concentraremos en confirmar la cota para el peor caso, el mejor caso y el comportamiento del algoritmo para casos aleatorios.

\begin{itemize}
\item El mejor caso se da cuando todos los androides están alineados en una misma fila, columna o diagonal, bajo este supuesto basta un solo kamehameha para destruirlos.
\item El peor caso se da cuando los androides están ubicados en una posición tal que cualquier semirrecta que atraviese a un par de ellos no atravesara a ninguno del resto por lo que necesitaremos una cantidad de kamehamehas igual a la mitad de los androides para destruirlos a todos. Un caso como el mencionado se da si ubicamos a los androides sobre los puntos de una parábola de una función como puede ser una cuadrática por ejemplo.
\item El otro caso se da en aquellos que no caen en ninguno de los mencionados anteriormente sino que los androides son ubicados aleatoriamente en el plano.
\end{itemize}

A continuación se muestran los resultados generados.

\begin{figure}[h]
	\begin{center}
%		\resizebox{\columnwidth}{!}{\input{imgs/p3-mediciones.png}}
	\end{center}
	\caption{Mediciones de peor caso, mejor caso y caso promedio}
\end{figure}



\newpage
\section{Problema 3: Kamehameha}
\subsection{Introducción}
\subsubsection{Explicación del problema}
El problema consiste en destruir a los androides del doctor Maki Gero mediante la menor cantidad posible de Kamehamehas lanzados por Goku.
A fin de resolver este problema modelamos a los androides como puntos en el plano y los Kamehameha como semirrectas en el mismo. Dicho es el problema se reduce a atravesar todos los puntos del plano con la menor cantidad de semirrectas.

\subsubsection{Ideas e implementación}
Para resolver el problema usaremos la técnica de backtracking. Mediante el mismo iremos eligiendo de a 2 elementos dentro de un vector con las posiciones de los androides y calculando la pendiente de la recta que los une nos fijaremos si atraviesa posiciones de otros androides en su recorrido. Marcaremos todos estos puntos como androides destruidos.
Una posible solución es encontrada cuando no quedan androides por destruir.
La estructura del algoritmo de backtracking es la siguiente:

\begin{algorithm}[h!]
\caption{Estructura del algoritmo de Backtracking}
\begin{algorithmic}[1]
	\Function{Backtrack}{}
	\If{no quedan mas androides por destruir}
		\State Mejor solución encontrada hasta ahora
		\State Guardo el vector de androides como mejor solución
	\Else
		\For{cada par de androides $a_1$, $a_2$}
            \If{$a_1$ y $a_2$ no estan destruidos}
                \If{la solución parcial es menor que la mejor solución +1}
                    \State Destruir los androides $a_1$, $a_2$
                    \State Destruir los androides en la semirrecta entre $a_1$ y $a_2$
                    \State Backtrack()
                    \State Restaurar los androides en la semirrecta entre $a_1$ y $a_2$
                    \State Restaurar los androides $a_1$, $a_2$                        
                \EndIf
            \EndIf
		\EndFor
	\EndIf
	\EndFunction
\end{algorithmic}
\end{algorithm}

\subsubsection{Podas}
En esta sección se explican las podas implementadas.

\begin{enumerate}
\item Una de las primeras podas que aplicamos es la de verificar si el androide no se encuentra ya destruido para no seguir por esa rama.
\item Aplicamos otra poda comparando que los 2 androides a unir con la semirecta no sean el mismo.
\item Aplicamos también una poda al comparar la solución parcial con la mejor solución encontrada hasta ese momento y solo seguimos en caso que sea posible mejorar la mejor solución encontrada hasta ese momento.
\end{enumerate}


\subsection{Correctitud}
Para probar la correctitud de este algoritmo hay qur probar dos cosas:
1.que cubre todos los puntos
2.que es el resultado minimo:


1.Esta demostracion es trivial, es parte del algoritmo  calcular que se cubran todos los puntos
2.lo hacemos por induccion sobre la cantidad de puntos.

La hipotesis inductiva es que para N puntos el algoritmo es minimo

Caso base:
numero de puntos = 1:
Esta demostracion es trivial. el algoritmo nos muestra que tira la genkidama en ese unico punto y no se pueden tirar menos genkidamas

Caso N+1:
Dada  la HI, si agregamos un punto mas respetando las condiciones nos quedarian tres casos:
1.O bien la solucion planteada para los N primeros puntos alcanza para que muera el N+1 enemigo, en ese caso seguiria siendo minimo
2.O bien si la genkidama se tiraba en el punto N, y tirarla en N+1 mataria a N, entonces  en vez de tirarse en N se tira en N+1, nuevamente dando minimo
3.O bien el punto esta lo suficientemente lejos del resto de los puntos como para que la unica forma de matar a ese enemigo sea tirandole una genkidama a ese punto y dicha genkidama no mate a nadie, en ese caso volveria a ser un minimo por que se requeriria si o si una genkidama mas para llegar a ese punto
\subsection{Complejidad}
\subsubsection{Pseudocódigo}

\subsubsection{Analisis}


\subsection{Analisis experimental}
Para la realización de los tests nos concentraremos en confirmar la cota para el peor caso, el mejor caso y el comportamiento del algoritmo para casos aleatorios.

\begin{itemize}
\item El mejor caso se da cuando todos los androides están alineados en una misma fila, columna o diagonal, bajo este supuesto basta un solo kamehameha para destruirlos.
\item El peor caso se da cuando los androides están ubicados en una posición tal que cualquier semirrecta que atraviese a un par de ellos no atravesara a ninguno del resto por lo que necesitaremos una cantidad de kamehamehas igual a la mitad de los androides para destruirlos a todos. Un caso como el mencionado se da si ubicamos a los androides sobre los puntos de una parábola de una función como puede ser una cuadrática por ejemplo.
\item El otro caso se da en aquellos que no caen en ninguno de los mencionados anteriormente sino que los androides son ubicados aleatoriamente en el plano.
\end{itemize}

A continuación se muestran los resultados generados.

\begin{figure}[h]
	\begin{center}
%		\resizebox{\columnwidth}{!}{\input{imgs/p3-mediciones.png}}
	\end{center}
	\caption{Mediciones de peor caso, mejor caso y caso promedio}
\end{figure}




\ref{LastPage}

\end{document}
